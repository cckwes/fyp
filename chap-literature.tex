\chapter{Literature Review}\label{chap:literature}
\section{Introduction}
In this chapter, the research paper of different researchers around the world will be reviewed. The research on tuning the PWM controller signal, modification of phase current signal for minimizing the torque ripple as well as driving strategy for cars and trains will be reviewed.

\section{Torque Ripple}
Park, Sung Jun \textit{et al.} \citep{00824132} found that the back EMF generated by each of the three windings are slightly different in shape and magnitude from each other. Since the back EMF for each winding is different, therefore different phase current should be apply to each of the three windings. By assuming the three stator windings are in Y-shape connection, the cogging torque and the reluctance torque component is negligible and the mutual torque is directly proportional to the phase current, phase current for each of the three stator windings should be seperately excited in phase with the back EMF to minimize the losses and maximizing the torque-per-ampere generation.

Kim, Tae-Sung \textit{et al.} \citep{00976016} shows that by using the rectangular shape phase current that changes to high at the flat part of the back-emf wave can minimize the torque ripple, but this method is too ideal to be used in practical condition. In this paper, another method which is the unipolar PWM is introduced but it has a slow dynamic response, making this method not feasible at reducing the torque for current control commutation. The proposed current control algorithm is to seperate the phase current that contains all harmonic components to each harmonical components and then transformed into stationary frame. Then, each stationary frame is added together and the output is the current command. After this process, the rectangular wave will appear more rectangular and hence reducing the torque ripple.

Nam, Ki-Yong \textit{et al.} \citep{01608454} suggested reducing the torque ripple by varying the input voltage to the PMBLDC. The idea of this paper is that by maintaining the current at a constant value, the torque ripple could be minimized. The method for reducing the torque ripple used in this paper is to supply varied input voltage during the freewheeling region.

Wael A. Salah \textit{et al.} \citep{7648} proposed a method which apply a modified PWM signal to the PMBLDC. The modified PWM signal used will delay the build up of current in the in-coming phase gradually at low speed region. At high speed region, it will speed up the build up of current which results in overcoming the tips and dips of current during phase current commutation which contributes to reduce in torque ripple.

Mohamed. A. Enany \textit{et al.} \citep{285} presented a method for improving the performance of BLDC with varying the switch-on and switch-off angle of the phase current. By advancing the switch-on and switch-off phase current, it enables current at each windings to reach to the maximum value earlier hence reducing the tips and dips of the current at the other winding which results in reducing torque ripple.

G. H. Jang and C. J. Lee \citep{08305} proposed a method to reduce the torque ripple through eliminating cogging torque by implementing a modified current wave form. The modified current wave form consists of main and auxiliary wave which the main wave is the conventional wave whereas the auxiliary wave generates a torque which has the same magnitude but opposite direction to the cogging torque.

Vanisri A. and Devarajan N. \citep{1450216} describe the design of a controller with minimize torque ripple which different than the conventional controller for PMBLDC motor by filter components. The methodology of the method used in this research is passing through the signal through an inductor-capacitor filter which filters out the high frequency waveform. The capacitor is selected in a way that it can charge and discharge effectively and the inductor is responsible for reduce the current pulsation hence reducing the torque ripple.

Leila Parsa and Lei Hao \citep{04435197} studied the effect of magnetization, winding distribution, skew angle and current angle on torque pulsation minimization. The switching instance has been calculated in a way that the reluctance torque is utilized in reducing the torque pulsation. It is also shown in the paper that by using the proper switching interval and applying suitable current waveform, the torque pulsation is reduced. 

\section{Driving Strategy}

