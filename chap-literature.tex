\chapter{Literature Review}\label{chap:literature}
\section{Introduction}
In this chapter, the research paper of different researchers around the world will be reviewed. The research on tuning the PWM controller signal, modification of phase current signal for minimizing the torque ripple as well as driving strategy for cars and trains will be reviewed.

\section{Torque Ripple}
Park, Sung Jun \textit{et al.} \citep{00824132} found that the back EMF generated by each of the three windings are slightly different in shape and magnitude from each other. Since the back EMF for each winding is different, therefore different phase current should be apply to each of the three windings. By assuming the three stator windings are in Y-shape connection, the cogging torque and the reluctance torque component is negligible and the mutual torque is directly proportional to the phase current, phase current for each of the three stator windings should be seperately excited in phase with the back EMF to minimize the losses and maximizing the torque-per-ampere generation.

Kim, Tae-Sung \textit{et al.} \citep{00976016} shows that by using the rectangular shape phase current that changes to high at the flat part of the back-emf wave can minimize the torque ripple, but this method is too ideal to be used in practical condition. In this paper, another method which is the unipolar PWM is introduced but it has a slow dynamic response, making this method not feasible at reducing the torque for current control commutation. The proposed current control algorithm is to seperate the phase current that contains all harmonic components to each harmonical components and then transformed into stationary frame. Then, each stationary frame is added together and the output is the current command. After this process, the rectangular wave will appear more rectangular and hence reducing the torque ripple.

Nam, Ki-Yong \textit{et al.} \citep{01608454} suggest reducing the torque ripple by varying the input voltage to the PMBLDC. The idea of this paper is that by maintaining the current at a constant value, the torque ripple could be minimized. The method for reducing the torque ripple used in this paper is to supply varied input voltage during the freewheeling region.

