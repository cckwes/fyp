\begin{MsAbstract}

Dalam Perlumbaan "Shell Eco-Marathon" kategori "Urban Concept", sebuah kereta yang serupa dengan kereta atas jalan dibina. Tertakluk kepada kategori yang disertai, yang dalam konteks ini kategori kereta elektrik "plug-in", motor electrik digunakan sebagai enjin kereta elektrik. Enjin kereta elektrik didirikan daripada pengawal dan sebuah "hub motor" yang disambung terus ke roda kereta, kedua-dua komponent tersebut adalah proprietari dan membukakan komponent tersebut dilarang sebab isu penajaan. Oleh itu, isyarat keluaran daripada pengawal iaitu voltan dan isyarat kelajuan telah dikenalpastikan untuk menubuhkan sebuah sistem pengukuran yang boleh digunakan untuk mengukur, mencatat dan memaparkan parameter tersebut. Sistem pengukuran tersebut ditubuhkan dengan menggunakan mikro-pengawal yang berjenama "Arduino", penghubung pemaparan ITDB02 dengan kad SD dan LCD skrin ITDB02-3.2WC untuk pemaparan. Selain daripada menggunakan sistem pengukuran untuk mengumpulkan data, perisian simulasi dinamik kenderaan yang berupaya mengsimulasikan rintangan rodaan, daya pengheretan, naik/turun cerun dan memandu serta mempercepatkan kenderaan dalam Sepang North Track dibina. Selepas itu, empat strategi untuk memandukan kereta tersebut dalam litar telah dibina bertujukan perjalanan paling jauh dengan tenaga yang paling kurang.  

\end{MsAbstract}
