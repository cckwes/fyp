\chapter{Result and Discussion}

\section{Introduction}
In this chapter, the result of signal identification will be displayed and explained. Furthermore, the result of the measurement system which includes the schematic of the system will be shown. After that, the vehicle simulation program and the result of 1 lap of simulation around Sepang North Track will be displayed and discussed. Finally, several strategies planned using the simulation software will be shown follow by discussion on which strategy is better and comparison between the strategies.

\section{Signal Identification}

\section{Measurement System}

\section{Vehicle Simulation}
The vehicle dynamics simulation software is built. As explained in last chapter, the simulation software has the ability to simulate the vehicle behaviour when driving around the Sepang North Track. As shown in figure \ref{im:vehicleSettings}, the software is initialized before the simulation can be started. The software takes a few parameter for initialization of the vehicle model which includes the vehicle driving wheel diameter, the total mass of the electric vehicle include the mass of the driver, the frontal area of the electric vehicle, the coefficient of rolling resistance and the coefficient of drag. The parameter for initializing the vehicle model is shown in table \ref{tb:vehicleModelParameter}.

\begin{figure}[htb]
	\centering
	\includegraphics[width=5in]{images/vehicle_settings.jpg}
	\caption{Vehicle parameter for initializing the vehicle model}
	\label{im:vehicleSettings}
\end{figure}

\begin{table}[htbp]
\begin{center}
\begin{tabular}{|c|c|}
\hline
\textbf{Parameter} & \textbf{Value} \\ \hline
Wheel Radius & 0.5 m \\ \hline
Total Vehicle Mass & 250 kg \\ \hline
Frontal Area & 1.4 $m^2$ \\ \hline
Crr & 0.016 \\ \hline
Cd & 0.7 \\ \hline
\end{tabular}
\end{center}
\caption{Parameters for building the electric vehicle model}
\label{tb:vehicleModelParameter}
\end{table}

The motor and track model is initialized at the second and third tab of the initialization dialog as shown if figure \ref{im:motorSettings} and figure \ref{im:trackSettings}. At the last tab, the displacement interval for iteration is set which has a range from 1m to 25m, which is shown in figure \ref{im:iterationStep}.

\begin{figure}[htb]
	\centering
	\includegraphics[width=5in]{images/motor_settings.jpg}
	\caption{Choosing the motor model}
	\label{im:motorSettings}
\end{figure}

\begin{figure}[htb]
	\centering
	\includegraphics[width=5in]{images/track_settings.jpg}
	\caption{Choosing the track model}
	\label{im:trackSettings}
\end{figure}

\begin{figure}[htb]
	\centering
	\includegraphics[width=5in]{images/iteration_step.jpg}
	\caption{Setting the displacement interval for iteration}
	\label{im:iterationStep}
\end{figure}

The simulation was run using "full throttle everywhere" strategy. The "full throttle everywhere" is the fundamental strategy for building the baseline data by applying full throttle over the entire circuit, which would give a maximum energy consumption result. The graph listed below is plot based on the simulation result:

\begin{itemize}
	\item{Speed and gradient versus displacement}
	\item{Power and gradient versus displacement}
	\item{Air drag versus displacement}
\end{itemize}

The graph of vehicle speed and track gradient versus the displacement is shown in figure \ref{im:0_1}. According to the graph, the maximum speed is 20.6m/s. The maximum speed is achieved at 1100m where the vehicle is driving downhill. The speed of the vehicle fluctuates between 20.6m/s and 12.6m/s from 1100m displacement onwards until end of the lap. The vehicle speed varies with the pattern of the gradient which it slows down when going up the hill because the torque generated is less than the total resistance torque excerted on the vehicle. The electric vehicle speeds up when going down the hill simply because the torque generated is excessive.

\begin{figure}[htb]
	\centering
	\includegraphics[width=6in]{images/0_1.jpg}
	\caption{Graph of speed and gradient versus displacement for "full throttle everywhere"}
	\label{im:0_1}
\end{figure}

Figure \ref{im:0_2} shows the power and gradient versus displacement. The power profile is the same as the speed profile as the torque generated by the electric motor is the same throughout the whole circuit which is constant 100N.m output. This is because the maximum torque output of the motor below 800RPM is 100N.m. The maximum power output of the electric vehicle is at around 1000m with a value of 4050W. 

\begin{figure}[htb]
	\centering
	\includegraphics[width=6in]{images/0_2.jpg}
	\caption{Graph of power and gradient versus displacement for "full throttle everywhere"}
	\label{im:0_2}
\end{figure}

Figure \ref{im:0_3} displays the graph of air drag versus displacement. Again, the air-drag profile is similar to the vehicle speed curve because the air-drag force is proportional to the square of the velocity of the vehicle. The maximum air drag occurs when the speed of the vehicle hits maximum with the magnitude of 120 N.m of torque needed for overcoming the air-drag.

\begin{figure}[htb]
	\centering
	\includegraphics[width=6in]{images/0_3.jpg}
	\caption{Graph of air drag versus displacement for "full throttle everywhere"}
	\label{im:0_3}
\end{figure}

Table \ref{tb:fullThrottleResult} shows the main result for the "Full Throttle Everywhere". As shown in the table, the total energy consumption as simulated is 560003J and the lap time is 187 seconds. The total time for 4 laps will be 748 seconds, with 10 seconds of stops between each lap and some time wasted, the time for completion for 1 attempt will be 788 seconds which is 13 minutes and 8 seconds. The mileage calculated from the energy consumption is 18 km/kWh. 

\begin{table}[htbp]
\begin{center}
\begin{tabular}{|c|c|}
\hline
\textbf{Result} & \textbf{Value} \\ \hline
Total Energy Consumption & 560003J \\ \hline
Total Time for 1 Lap & 186.981s \\ \hline
Mileage & 18 km/kWh \\ \hline
\end{tabular}
\end{center}
\caption{Main result for "Full Throttle Everywhere" }
\label{tb:fullThrottleResult}
\end{table} \clearpage

\section{Strategies}
There are 3 strategies composed for improving the mileage of the electric vehicle on the Sepang North Track.

\subsection{Preset Strategy 1}

The "Full Throttle Everywhere" strategy listed at the previous section is not effective because the torque generated by the electric motor is excessive at most of the part of the circuit. Therefore, a better strategy is produced which is the "Preset Strategy 1". The "Preset Strategy 1" reduces the energy consumption of the electric vehicle by turning off the electric motor so that the electric vehicle cruise down the hill when the gradient is less than 0\textdegree \ . By using the simulation software, the result of "Preset Strategy 1" is shown at the following graphs.

Figure \ref{im:1_1} shows the graph of speed and gradient versus displacement for "Preset Strategy 1". The speed profile of the electric vehicle shown in this graph is different with the speed curve in figure \ref{im:0_1} since the vehicle is set to cruise when the gradient is negative. The maximum speed of the vehicle is less than the maximum speed for "FUll Throttle Everywhere" strategy which has a value of 15.7m/s and happens at the end of starting straight. The speed of the vehicle drops gradually when going downhill because the gravitational force is insufficient for countering the rolling resistance and air drag. The vehicle pick up speed when the gradient of the track is less than 3\textdegree and started to slow down when the gradient is higher due to the inadequate torque generated by the electric motor.

\begin{figure}[htb]
	\centering
	\includegraphics[width=6in]{images/1_1.jpg}
	\caption{Graph of Speed and Gradient versus displacement for "Preset Strategy 1"}
	\label{im:1_1}
\end{figure}

Figure \ref{im:1_2} shows the graph of power and gradient versus the displacement for "Preset Strategy 1". The power curve shows a huge rise at the beginning straight. The power output of the electric vehicle is 0 when cruising down and the power output surge again when the vehicle is driving on the flat road or up the hill. Since the speed of the vehicle is less than the speed of the vehicle in "Full Throttle Everywhere" strategy, the maximum output also drops from over 4kW for the previous strategy to around 3.1kW for the current strategy.

\begin{figure}[htb]
	\centering
	\includegraphics[width=6in]{images/1_2.jpg}
	\caption{Graph of Power and Gradient versus displacement for "Preset Strategy 1"}
	\label{im:1_2}
\end{figure}

Since the maximum vehicle speed is reduced by 23.8\% , the maximum torque need to be supplied by the electric motor for counter air drag force reduces from 120 N.m to 69.5 N.m, which is a huge 42\% improvement as shown in figure \ref{im:1_3}.

\begin{figure}[htb]
	\centering
	\includegraphics[width=6in]{images/1_3.jpg}
	\caption{Graph of air drag versus displacement for "Preset Strategy 1"}
	\label{im:1_3}
\end{figure}

Table \ref{tb:preset1Result} shows the overall result simulated using "Preset Strategy 1". The total energy consumption is reduced to 365004J but the lap time is increased to 247 seconds. The time for 1 attempt is increases to 1026 seconds which is 17 minutes and 6 seconds which is within the time limit. The mileage using Preset Strategy 1 is improved to 27.6 km/KWh.

\begin{table}[htbp]
\begin{center}
\begin{tabular}{|c|c|}
\hline
\textbf{Result} & \textbf{Value} \\ \hline
Total Energy Consumption & 365004J \\ \hline
Total Time for 1 Lap & 246.554s \\ \hline
Mileage & 27.6 km/kWh \\ \hline
\end{tabular}
\end{center}
\caption{Main result for "Preset Strategy 1" }
\label{tb:preset1Result}
\end{table} \clearpage

\subsection{Preset Strategy 2}
Since the "Preset Strategy 1" waste a lot of energy by accelerating through the ending straight and the speed of the vehicle is stil reducable for reducing the effect of air-drag, "Preset Strategy 2" is set so that the vehicle cruise to the end of a lap and the speed is controlled within 14m/s and the vehicle started to cruise at 2650m displacement onwards until the start/finish line.

Figure 

\begin{figure}[htb]
	\centering
	\includegraphics[width=6in]{images/2_1.jpg}
	\caption{Graph of Speed and Gradient versus displacement for "Preset Strategy 2"}
	\label{im:2_1}
\end{figure}

\begin{figure}[htb]
	\centering
	\includegraphics[width=6in]{images/2_2.jpg}
	\caption{Graph of Power and Gradient versus displacement for "Preset Strategy 2"}
	\label{im:2_2}
\end{figure}

\begin{figure}[htb]
	\centering
	\includegraphics[width=6in]{images/2_3.jpg}
	\caption{Graph of air drag versus displacement for "Preset Strategy 2"}
	\label{im:2_3}
\end{figure}

\section{Summary}
