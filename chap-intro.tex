\chapter{Introduction}\label{chap:intro}
\section{Background}

Electric motor can be classified into two major categories which are the DC electric motor and AC electric motor. An AC motor is an electric motor driven by alternating current whereas a DC motor is driven by direct current. There are various types of AC motor which includes induction motor, synchronous motor, eddy current motor and etc. The DC electric motor includes permanent magnet brushed motor, permanent magnet brushless motor, switched reluctance motor and etc.

Electric motor is used in many application which includes in machine for driving the pulley and belts, the conveyor belt, in drilling and lathe machine. Apart from the heavy industry, electric motor is used in home appliances for powering the washing machine, fan, blower of air-conditioner and blender machine. Moreover, electric motor is also used in automobile industry as the starter motor for firing up the internal combustion engine of cars and trucks and last but not least, as the drive train for electric vehicle.

The PMBLDC is a synchronous motor. In other words, the frequency of the magnetic field generated at the stator and the rotor is the same. PMBLDC comes in single-phase, 2-phase and 3-phase configuration which the 3-phase configuration is the most popular among the three. There are basically two major components inside a PMBLDC motor which is the stator and the rotor. The stator of a PMBLDC motor is made up of 
a series of laminated steel with wire windings around it. The rotor is build up of permanent magnet that has at least 2 poles.

Unlike brushed motor, PMBLDC does not have brushes for comutation, instead a controller is needed for controlling the rotation of the PMBLDC by sending out AC signal to the PMBLDC. There are two types of AC signal sent to the PMBLDC for controlling the motor which are the Trapezoidal type and the Sinusoidal type which depends on the winding of the stator.

In order for the controller to send out the correct signal, the position of the rotor must be sent to the controller so that a sequence of AC signal can be generated which energized the winding of the stator for rotating the motor. Hall effect sensors is used as the rotor position detection sensor which has an analog signal output. When the magnetic field is detected by the hall effect sensor, the voltage output will be changes from lowest to the highest or vice versa depending on the circuit configuration. Normally there are three hall effect sensor mounted on the stator of PMBLDC motor which are 60\textdegree or 120\textdegree apart depending on the number of poles and the comutation sequence required.

